\documentclass{article}
\usepackage[english]{babel}
\usepackage{listings}
\usepackage{hyperref}


\title{Exercise book, Excersise 1, TDT4195}
\author{Kim Rune Solstad}

\begin{document}
\lstset{language=Matlab}

\maketitle

\section*{3 Basic Image Manipulation}
\subsection*{Task 1}
\lstinputlisting{Task3.m}
\subsection*{Task 2}
The pixels in the flat field image are assigned a number slightly above 1 if less sensitive and slightly below 1 if more sensitive. This means that the corrections are given in ratios instead of hard values. Therefore we need to divide with the flat-field-matrix and not subtract.

\section*{4 Filtering}
\subsection*{Task 1}

\begin{tabular}{|c|*{16}{c|}r}
0&0&0&0&0&0& 1&2&3&4&5&4&3&2&1&0&0
\end{tabular}

\section*{5 Noise}
\lstinputlisting{Task5.m}

\section*{6 Aliasing}
\lstinputlisting{Task6.m}
For some reason, I end up getting three images in a row. Using the downsampling function requires me to rotate the image befoure renaming the collumns.


\section*{Ref}
cynanogen.com/help/maximdl/Flat-Field\_Frame\_Calibration.htm
\end{document}

